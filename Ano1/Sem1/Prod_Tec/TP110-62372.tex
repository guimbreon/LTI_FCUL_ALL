\documentclass{report}
\usepackage{graphicx}
\usepackage{geometry}
\usepackage{quoting}
\usepackage{float}
\usepackage{caption}
\usepackage{pgfplots}
\usepackage{booktabs}
\usepackage{tabularx}
\usepackage{titlepic}
\usepackage{calc}

\geometry{
  top=1.5cm,
  bottom=1.5cm,
  left=1.9cm,
  right=1.9cm
}

\begin{document}
\titlepic{\includegraphics[scale=0.6]{imagemCapa.jpg}}
\title{Viva-Melhor\\Licenciatura em Tecnologias de Informação \\ Turma: TP110}
\author{Guilherme Alexandre Cardoso Soares, nº 62372}
\date{1º sem 2023-2024}
\maketitle

\tableofcontents
\setcounter{page}{1}
\pagenumbering{roman}

\listoftables
\listoffigures

\newpage
\pagenumbering{arabic}
\setcounter{page}{4}

\chapter{Introdução}

Os seguros de saúde desempenham um papel crucial na sociedade, atuando como contratos financeiros que asseguram cobertura financeira e assistência médica em momentos de necessidade. Sua principal finalidade é garantir que as pessoas tenham acesso a cuidados de saúde adequados quando mais precisam. Os seguros de saúde são, portanto, acordos legalmente vinculativos entre duas ou mais partes, estabelecendo obrigações e direitos específicos, conforme definido no Artigo 1 da Seção no Decreto-Lei n.º 72/2008.

\begin{quoting}
  Seção 1, Artigo 1.º - Princípio geral

  O contrato de seguro rege-se pelo princípio da liberdade contratual, tendo carácter supletivo as regras constantes do presente regime, com os limites indicados na presente secção e os decorrentes da lei geral.
\end{quoting}

\begin{minipage}{0.45\textwidth}
  \textbf{Vantagens dos Seguros de Saúde:}
  \begin{itemize}
    \item Acesso a cuidados médicos
    \item Cobertura financeira
    \item Atendimento rápido
    \item Prevenção de doenças
    \item Paz de espírito
  \end{itemize}
\end{minipage}%
\begin{minipage}{0.45\textwidth}
  \textbf{Desvantagens dos Seguros de Saúde:}
  \begin{itemize}
    \item Custos mensais
    \item Limitações na cobertura
    \item Dificuldade em encontrar médicos da rede
    \item Carências e restrições
    \item Burocracia
    \item Aumentos de prêmios
  \end{itemize}
\end{minipage}

\chapter{Caso de Estudo - Viva Melhor}

O presente estudo analisa as opções de seguros de saúde disponíveis para o senhor Nelson, representado pelo número de aluno 62372. Em conformidade com a Declaração Universal dos Direitos Humanos, que preconiza o direito à saúde, Portugal oferece um sistema público, o Serviço Nacional de Saúde (SNS). Contudo, devido à demanda crescente, surgem seguros privados como alternativa. O senhor Nelson busca a melhor opção para seu agregado familiar entre três planos de seguro (A, B e C), considerando o número de atos médicos diurnos e noturnos esperados.

\begin{table}[H]
  \centering
  \begin{tabular}{|c|c|c|c|}
    \hline
    & Seguro A & Seguro B & Seguro C \\
    \hline
    Anuidade & 37 & 72 & 62 \\ \hline
    Tarifa de ato médico diurno & 11 & 13 & 10 \\ \hline
    Tarifa de ato médico noturno & 3 & 6 & 5 \\ \hline
    Atos médicos oferecidos com a anuidade & 10 & 15 & 20 \\
    \hline
  \end{tabular}
  \caption{Tabela com o preçario dos Seguros}
\end{table}

\begin{table}[h]
  \centering
  \begin{tabular}{|c|c|c|c|}
    \hline
    Nº Atos & Seguro A & Seguro B & Seguro C \\
    \hline
    10 & 37 & 72 & 62 \\ \hline
    20 & 115 & 123 & 62 \\ \hline
    30 & 193 & 225 & 142 \\ \hline
    40 & 271 & 327 & 222 \\ \hline
    50 & 349 & 429 & 302 \\ \hline
    60 & 427 & 531 & 382 \\ \hline
    70 & 505 & 633 & 462 \\ \hline
    80 & 583 & 735 & 542 \\ \hline
    90 & 661 & 837 & 622 \\ \hline
    100 & 739 & 939 & 702 \\ \hline
    110 & 817 & 1041 & 782 \\
    \hline
  \end{tabular}
  \caption{Tabela de Valores por dados Atos}
  \label{tab:valores}
\end{table}

\begin{figure}[H]
  \begin{center}
    \begin{tikzpicture}
      \begin{axis}[
          xlabel={Nº Atos Médicos Anuais},
          ylabel={Custo Total},
          legend style={at={(0.5,-0.2)},anchor=north},
        ]
        \addplot[smooth,mark=*,blue] table [x=N, y=SeguroA, col sep=comma] {dados.csv};
        \addlegendentry{Seguro A}

        \addplot[smooth,mark=square*,red] table [x=N, y=SeguroB, col sep=comma] {dados.csv};
        \addlegendentry{Seguro B}

        \addplot[smooth,mark=triangle*,green] table [x=N, y=SeguroC, col sep=comma] {dados.csv};
        \addlegendentry{Seguro C}
      \end{axis}
    \end{tikzpicture}
  \end{center}
  \caption{Gráfico comparativo dos Seguros de Saúde.}
  \label{fig:comparison}
\end{figure}

\chapter{Conclusão}

Conforme evidenciado no gráfico apresentado na Figura~\ref{fig:comparison}, o qual reflete os dados contidos na Tabela~\ref{tab:valores}, é possível concluir que, em termos gerais, o Seguro C destaca-se positivamente em comparação com os outros dois seguros analisados. Apesar de apresentar uma anuidade mais elevada, este seguro oferece uma gama mais abrangente de benefícios. Destaca-se por oferecer uma tarifa mais vantajosa para atos médicos diurnos (10), ser a segunda opção mais econômica para atos médicos noturnos (5) e se destacar como a opção que proporciona o maior número de atos médicos incluídos na anuidade (20).

Podemos então, comparar com a média de preços de 2020 que era cerca de 338 Euros, ou seja, cerca de 30 euros mensais (sendo essa a anuidade), vamos considerar então, que o Sr. Nelson queira ter 30 visitas ao médico, terá de pagar ao todo 142 Euros, ou seja, cerca de 14 euros mensalmente (ECO, 2021, 10 de fevereiro).

Temos ainda que comparar com os valores dados por P. Andersson (2021, Março 5) no jornal e-konomista.pt, que, por exemplo, 500 euros por ano dariam para pagar cerca de 5 consultas de especialidade de 80 euros sem qualquer desconto, e ainda lhe sobravam 100 euros para exames e alguns medicamentos.

E tendo em conta os diferentes pontos de vista sobre o custo-benefício dos seguros de saúde, é essencial que o Sr. Nelson leve em consideração suas próprias necessidades e preferências ao escolher um plano.

Além disso, ao analisar a oferta de seguros de saúde, o Sr. Nelson deve levar em consideração não apenas o custo da anuidade, mas também os benefícios oferecidos, como cobertura para atos médicos diurnos e noturnos, número de consultas incluídas na anuidade, entre outros fatores.

É crucial que o Sr. Nelson avalie cuidadosamente as condições, franquias e exclusões de cada plano, garantindo que escolha aquele que melhor atenda às suas necessidades específicas. Além disso, ele pode considerar a possibilidade de consultar mediadores de seguros para obter simulações e comparar ofertas de diferentes seguradoras.

Em síntese, minha recomendação para o Sr. Nelson é o seguro A. No entanto, é crucial que ele leve em consideração os pontos discutidos anteriormente para tomar uma decisão informada sobre qual seguro melhor atende às suas necessidades específicas.

\chapter{Referências}

\begin{itemize}
  \item Decreto-Lei n.º 72/2008, de 16 de abril. Regime jurídico do contrato de seguro. Diário da República, Capítulo 1, Secção 1, Artigo 1 (Conteúdo típico).
  \item Andersson, P. (2021, Março 5). Qual é o melhor seguro de saúde em Portugal? Ekonomista. https://www.e-konomista.pt/qual-o-melhor-seguro-de-saude-em-portugal/
  \item ECO. (2021, Fevereiro 10). Portugueses com seguros de saúde já são quase 3 milhões. ECO. https://eco.sapo.pt/2021/02/10/portugueses-com-seguros-de-saude-ja-sao-quase-3-milhoes/
\end{itemize}
\end{document}
